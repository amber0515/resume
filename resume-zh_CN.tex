% !TEX TS-program = xelatex
% !TEX encoding = UTF-8 Unicode
% !Mode:: "TeX:UTF-8"

\documentclass{resume}
\usepackage{zh_CN-Adobefonts_external} % Simplified Chinese Support using external fonts (./fonts/zh_CN-Adobe/)
%\usepackage{zh_CN-Adobefonts_internal} % Simplified Chinese Support using system fonts
\usepackage{linespacing_fix} % disable extra space before next section
\usepackage{cite}

\begin{document}
\pagenumbering{gobble} % suppress displaying page number

\name{安澜霆 Amber}

% {E-mail}{mobilephone}{homepage}
% be careful of _ in emaill address
\contactInfo{(+86) 156-1118-7755}{532389813@qq.com}
% {E-mail}{mobilephone}
% keep the last empty braces!
%\contactInfo{xxx@yuanbin.me}{(+86) 131-221-87xxx}{}

\section{教育背景}
\datedsubsection{\textbf{清华大学},工业工程,\textit{在读硕士}}{2022.9 - 2025.6}
\datedsubsection{\textbf{复旦大学},软件工程,\textit{本科}}{2015.9 - 2019.6}
\ \textbf{复旦大学三等奖奖学金},团学联体育部副部长

%***********职业经历*********************
\section{职业经历}

\datedsubsection{\textbf{OKX Group},Web3合约组,智能合约开发工程师}{2022.04 - 至今}
\begin{itemize}[parsep=0.5ex]
  \item 负责 NFT 聚合器与二级市场智能合约的架构设计、流动性接入、代码开发相关工作;
  \item 负责 NFT Launchpad, Campaign 与创作页的合约架构设计与开发;
  \item 负责 web3 行业内新项目与技术调研并推进产品化应用;
\end{itemize}

\datedsubsection{\textbf{中国外运},运营及数字化部,技术经理}{2021.10 - 2022.03}
\begin{itemize}[parsep=0.5ex]
  \item 与各部门协调,为各部门提供信息技术支持并对全公司的信息资源进行管理和控制;
  \item 根据企业发展战略和信息化战略要求,负责企业内外部信息资源开发利用;
  \item 参与公司信息化建设的总体规划及架构设计,建立企业信息化管理制度和标准规范;
\end{itemize}

\datedsubsection{\textbf{中国外运},工程技术部,Java开发工程师}{2019.08 - 2021.10}
一、HS 编码智能归类系统
\begin{itemize}[parsep=0.5ex]
  \item 基于springboot框架,elasticsearch,mysql开发HS编码智能归类系统。其中负责了代码架构搭建、基本功能开发、索引设计、数据库设计以及功能优化。使用nginx部署前端项目,jenkins 构建 CI/CD 流程,将项目构建为docker进行部署。
\end{itemize}
二、智慧工程物流系统
\begin{itemize}[parsep=0.5ex]
  \item 基于springboot框架开发智慧工程物流系统。其中负责了基本功能开发、数据库设计以及功能优化。使用 elasticsearch、logstash、kibana 搭建系统日志管理系统,保证系统的管理要求。
\end{itemize}

\datedsubsection{\textbf{上海复星杏脉科技},QT 开发工程师}{2019.02 - 2019.05}
一、8孔呼吸道病毒数据显示桌面挂件工具
\begin{itemize}[parsep=0.5ex]
  \item 使用 QT 开发平台,开发8孔呼吸道病毒数据显示桌面挂件工具。其中主要实现:多进程实现 QT c++与 python 算法(tensorflow-gpu)联调、多线程实现 QTc++ 与 python 功能函数联调。
\end{itemize}

%***********项目经历*********************
\section{项目经历}

\datedsubsection{\textbf{NFT Lending},Solidity合约工程师}{2021.06 - 至今}
\begin{itemize}[parsep=0.5ex]
  \item 基于 solidity、Openzeppelin 实现 Dapp 的智能合约签名验证、收益计算等合约功能;
  \item 使用 Hardhat 对合约功能进行单元测试和流程测试,并将合约部署至 Rinkeby;
  \item 编写 subgraph mapping script,将 event 数据索引至项目的 the graph 节点;
  \item 使用 slither 检查合约代码中的 bad smell 并进行修复;
\end{itemize}
\end{document}
